\section*{附录}
\addcontentsline{toc}{section}{附录}
\renewcommand\thesubsection{\Alph{subsection}}
\subsection{TRAN 分析诱导 AC 分析方程}\label{appendix:TRAN-to-AC-equation}
求解瞬态方程 \ref{eq:flat-equation} 及直流稳态方程 $\bm{F}(\bm{x},\bm{p})=\bm{0}$ 时,
Newton–Raphson 方法需反复计算方程余项 $\bm{Q},\bm{F}\in\mr^N$ 及 Jacobian 矩阵
$\nabla_{\bm{x}}\bm{Q},\nabla_{\bm{x}}\bm{F}\in\mr^{N\times N}$。
对交流小信号分析而言,考虑在稳态解附近对 $\bm{x},\bm{p}$ 作扰动
$\delta\bm{x},\delta\bm{p}$ 并线性化 \ref{eq:flat-equation} 得到
\[
  \nabla_{\bm{x}}\bm{Q}\cdot\dot{\delta}\bm{x}
  +\nabla_{\bm{p}}\bm{Q}\cdot\dot{\delta}\bm{p}
  +\nabla_{\bm{x}}\bm{F}\cdot\delta\bm{x}
  +\nabla_{\bm{p}}\bm{F}\cdot\delta\bm{p}=\bm{0}
\]
若假设$\delta\bm{x},\delta\bm{p}$是角频率为 $\omega$ 的小信号
$\bm{\epsilon}_{\bm{x}}\cdot e^{i\omega t},\bm{\epsilon}_{\bm{p}}\cdot e^{i\omega t}$,
则得到交流小信号分析的线性方程组
\begin{equation}\label{eq:flat-ac-equation}
  (i\omega\cdot\nabla_{\bm{x}}\bm{Q}+\nabla_{\bm{x}}\bm{F})\cdot\bm{\epsilon}_{\bm{x}}
  = (i\omega\cdot\nabla_{\bm{p}}\bm{Q}+\nabla_{\bm{p}}\bm{F})\cdot\bm{\epsilon}_{\bm{p}}
\end{equation}

\subsection{CMOS 子电路}\label{appendix:mos-subckt}
\begin{lstlisting}[language=json,numbers=none,
caption={CMOS 子电路},label=lst:mos-subckt]
"NMOSTYPE":{
  "ExternalNodes":["gate","source","drain","bulk"],
  "InputParams":["MosL","MosW"],
  "InternalNodes":[],
  "SubModel":{
    "Analysis":["DC","TRAN"],
    "ModelLoader":"SimInfo->lut.MosLookup(\"NMOSTYPE\",
      /path/to/data; SimInfo=SimInfo)",
    "IntrinsicParams":
      ["ID","GDS","CDD","CSS","CGG","CGS","CGD","GM","GMB"]
  },
  "Schematic":{
    "ids":{
      "MasterName":"ICS",
      "ExternalNodes":{"input":"source","output":"drain"},
      "InputParams":{"dc":"ID","ac":0}
    },
    "template":{
      "MasterName":"MosSmallSignalTemplate",
      "ExternalNodes":{
        "gate":"gate","source":"source",
        "drain":"drain","bulk":"bulk"
      },
      "InputParams":{
        "GDS":"GDS","CDD":"CDD","CSS":"CSS","CGG":"CGG",
        "CGS":"CGS","CGD":"CGD","GM":"GM","GMB":"GMB"
      }
    }
  }
}
\end{lstlisting}
\begin{lstlisting}[language=json,numbers=none,
caption={MosSmallSignalTemplate:小信号等效电路分解},label=lst:mos-small-signal-subckt]
"MosSmallSignalTemplate":{
    "ExternalNodes":["gate","source","drain","bulk"],
    "InputParams":["GDS","CDD","CSS","CGG","CGS","CGD","GM","GMB"],
    "InternalNodes":[],
    "Schematic":{
        "infr":{
            "MasterName":"resistor",
            "ExternalNodes":{"left":"drain","right":"source"},
            "InputParams":{"resistance":1e1000}
        },
        "gds":{
            "MasterName":"ACVCCS",
            "ExternalNodes":{"left":"drain","right":"source","input":"drain","output":"source"},
            "InputParams":{"MF":"GDS"}
        },
        "cdd":{
            "MasterName":"capacitor",
            "ExternalNodes":{"input":"drain","output":"bulk"},
            "InputParams":{"capacitance":"CDD"}
        },
        "css":{
            "MasterName":"capacitor",
            "ExternalNodes":{"input":"source","output":"bulk"},
            "InputParams":{"capacitance":"CSS"}
        },
        "cgg":{
            "MasterName":"capacitor",
            "ExternalNodes":{"input":"gate","output":"bulk"},
            "InputParams":{"capacitance":"CGG"}
        },
        "cgs":{
            "MasterName":"capacitor",
            "ExternalNodes":{"input":"gate","output":"source"},
            "InputParams":{"capacitance":"CGS"}
        },
        "cgd":{
            "MasterName":"capacitor",
            "ExternalNodes":{"input":"gate","output":"drain"},
            "InputParams":{"capacitance":"CGD"}
        },
        "gm":{
            "MasterName":"ACVCCS",
            "ExternalNodes":{
                "left":"gate","right":"source","input":"drain","output":"source"
            },
            "InputParams":{"MF":"GM"}
        },
        "gmb":{
            "MasterName":"ACVCCS",
            "ExternalNodes":{
                "left":"bulk","right":"source","input":"drain","output":"source"
            },
            "InputParams":{"MF":"GMB"}
        }
    }
}
\end{lstlisting}

\subsection{线性方程组解的梯度反传}\label{appendix:inv-linear-equation-grad}
考虑关于 $\bm{v}$ 的实线性方程组 $\bm{A}(\bm{x})\bm{v}=\bm{b}(\bm{x})$,
其中 $\bm{A},\bm{b}$ 非线性依赖于 $\bm{x}$ 为稀疏矩阵/向量,且
$\nabla_{\bm{x}}\bm{A},\nabla_{\bm{x}}\bm{b}$
均可计算,则 $\bm{v}$ 也将非线性依赖于 $\bm{x}$,事实上,对方程组作微分可得
\[
  (\bm{A}+\nabla_{\bm{x}}\bm{A}\cdot\ud\bm{x})\cdot
  (\bm{v}+\nabla_{\bm{x}}\bm{v}\cdot\ud\bm{x})
  =\bm{b}+\nabla_{\bm{x}}\bm{b}\cdot\ud\bm{x}
\]
舍弃0阶及2阶项,可得下式对任意 $\ud\bm{x}$ 成立
\begin{equation}\label{eq:inv-full-differentiation}
  \nabla_{\bm{x}}\bm{A}\cdot\ud\bm{x}\cdot\bm{v}
  +\bm{A}\cdot\nabla_{\bm{x}}\bm{v}\cdot\ud\bm{x}
  =\nabla_{\bm{x}}\bm{b}\cdot\ud\bm{x},
\end{equation}
假设某个损失函数 $l(\bm{v})$ 可关于 $\bm{v}$ 求梯度 $\nabla_{\bm{v}}l$,
需将梯度反传到 $\bm{x}$,即需求解
$\nabla_{\bm{x}}l=\nabla_{\bm{v}}l\cdot\nabla_{\bm{x}}\bm{v}$。
我们无需真的计算 $\nabla_{\bm{x}}\bm{v}$ 并存下来(这可能是稠密的),
事实上由 Eq.\ref{eq:inv-full-differentiation} 有
\[
  \begin{split}
  & \nabla_{\bm{x}}\bm{v}\cdot\ud\bm{x}
  = \bm{A}^{-1}\cdot\big(\nabla_{\bm{x}}\bm{b}\cdot\ud\bm{x}-
  \nabla_{\bm{x}}\bm{A}\cdot\ud\bm{x}\cdot\bm{v}\big),\forall \ud\bm{x}, \\
  \Rightarrow
    & \nabla_{\bm{v}}l\cdot\nabla_{\bm{x}}\bm{v}\cdot\ud\bm{x}
    = (\nabla_{\bm{v}}l\cdot\bm{A}^{-1})\cdot\big(\nabla_{\bm{x}}\bm{b}\cdot\ud\bm{x}-
  \nabla_{\bm{x}}\bm{A}\cdot\ud\bm{x}\cdot\bm{v}\big),\forall \ud\bm{x}, \\
  \end{split}
\]
因此,为计算 $\nabla_{\bm{v}}l\cdot\nabla_{\bm{x}}\bm{v}$,
只需提前作一次稀疏矩阵线性方程组求解 $\nabla_{\bm{v}}l\cdot\bm{A}^{-1}$,
再对 $\ud\bm{x}$ 各位置逐个置为1,其余位置置0,即可得到
$\nabla_{\bm{x}}l=\nabla_{\bm{v}}l\cdot\nabla_{\bm{x}}\bm{v}$。

对于复数线性方程组情形,也可采用 Wirtinger 导数进行类似讨论。
