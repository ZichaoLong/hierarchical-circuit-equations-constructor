% !TEX TS-program = xelatex
% !TEX encoding = UTF-8
%
%
\documentclass{article}

\usepackage{arxiv}

\usepackage[utf8]{inputenc} % allow utf-8 input
\usepackage[T1]{fontenc}    % use 8-bit T1 fonts
\usepackage{hyperref}       % hyperlinks
\usepackage{url}            % simple URL typesetting
\usepackage{booktabs}       % professional-quality tables
\usepackage{amsfonts}       % blackboard math symbols
\usepackage{nicefrac}       % compact symbols for 1/2, etc.
\usepackage{microtype}      % microtypography


\usepackage{amsmath,amssymb,amsthm,listings,xcolor}
\newtheorem{assumption}{Assumption}
 
\colorlet{punct}{red!60!black}
\definecolor{background}{HTML}{EEEEEE}
\definecolor{delim}{RGB}{20,105,176}
\colorlet{numb}{magenta!60!black}
\lstdefinelanguage{json}{
    basicstyle=\normalfont\ttfamily,
    numbers=left,
    numberstyle=\scriptsize,
    stepnumber=1,
    numbersep=8pt,
    showstringspaces=false,
    breaklines=true,
    frame=lines,
    backgroundcolor=\color{background},
    literate=
     *{0}{{{\color{numb}0}}}{1}
      {1}{{{\color{numb}1}}}{1}
      {2}{{{\color{numb}2}}}{1}
      {3}{{{\color{numb}3}}}{1}
      {4}{{{\color{numb}4}}}{1}
      {5}{{{\color{numb}5}}}{1}
      {6}{{{\color{numb}6}}}{1}
      {7}{{{\color{numb}7}}}{1}
      {8}{{{\color{numb}8}}}{1}
      {9}{{{\color{numb}9}}}{1}
      {:}{{{\color{punct}{:}}}}{1}
      {,}{{{\color{punct}{,}}}}{1}
      {\{}{{{\color{delim}{\{}}}}{1}
      {\}}{{{\color{delim}{\}}}}}{1}
      {[}{{{\color{delim}{[}}}}{1}
      {]}{{{\color{delim}{]}}}}{1},
}
\definecolor{capri}{rgb}{0.0, 0.75, 1.0}


\usepackage{graphicx,wrapfig,multirow,diagbox,caption,subcaption,enumitem}
\usepackage{verbatim,algpseudocode}
\usepackage[boxruled]{algorithm2e}
\renewcommand{\listalgorithmcfname}{Algorithm List}
\renewcommand{\algorithmcfname}{Algorithm}
\usepackage[numbers,sort,compress]{natbib}
\usepackage{bm,mathrsfs}
\newcommand{\lookup}{\text{lookup}}
\newcommand{\net}{\text{net}}
\newcommand{\device}{\text{device}}

\newcommand{\vp}{\varphi}
\newcommand{\al}{\alpha}
\newcommand{\be}{\beta}
\newcommand{\ti}{\tilde}
\newcommand{\ve}{\varepsilon}
\newcommand{\de}{\delta}
\newcommand{\na}{\nabla}
\newcommand{\pd}{\partial}
\newcommand{\ud}{\mathrm{d}}
\newcommand{\mr}{\mathbb{R}}
\newcommand{\ms}{\mathbb{S}}
\newcommand{\mz}{\mathbb{Z}}
\newcommand{\mn}{\mathbb{N}}
\newcommand{\mc}{\mathbb{C}}
\newcommand{\one}{\textbf{1}}
\DeclareMathOperator*{\st}{s.t.}

\title{Computational Graph Representation of Equations System Constructors in Hierarchical Circuit Simulation}
\author{Zichao Long
\And
Lin Li
\And
Lei Han
\And
Xianglong Meng
\And
Chongjun Ding
\And
Ruiyan Li
\And
Wu Jiang
\And
Fuchen Ding
\And
Jiaqing Yue
\And
Zhichao Li
\And
Yisheng Hu
\And
Ding Li
\And
Heng Liao
\thanks{Corresponding author}
\AND
\normalfont{HiSilicon}
}
\begin{document}
\maketitle
\begin{abstract}
  Equations system constructors of hierarchical circuits play a central role in
  device modeling, nonlinear equations solving, and circuit design automation.
  However, existing constructors present limitations in applications to different
  extents. For example, the costs of developing and reusing device models ---
  especially coarse-grained equivalent models of circuit modules ---
  remain high while parameter sensitivity analysis is complex and inefficient.
  % method % avoid introducing background
  Inspired by differentiable programming and
  leveraging the ecosystem benefits of open-source software, we propose
  an equations system constructor using the computational graph representation,
  along with its JSON format netlist, to address these limitations.
  This representation allows for runtime dependencies between signals and
  subcircuit/device parameters. The proposed method streamlines the model
  development process and facilitates end-to-end computation of gradients of
  equations remainders with respect to parameters.
  % overall idea
  This paper discusses in detail the overarching concept of hierarchical
  subcircuit/device decomposition and nested invocation by drawing parallels to
  functions in programming languages, and introduces
  rules for parameters passing and gradient propagation across
  hierarchical circuit modules.
  % examples and conclusions
  The presented numerical examples, including (1) an uncoupled CMOS model
  representation using "equivalent circuit decomposition+dynamic parameters"
  and (2) operational amplifier (OpAmp) auto device sizing, have demonstrated
  that the proposed method supports circuit simulation and design and
  particularly subcircuit modeling with improved efficiency, simplicity,
  and decoupling compared to existing techniques.
  % \\
  % {\bf{Keywords}}: hierarchical circuit simulation, SPICE, computational graph,
  % circuit static parameters, runtime variables, automatic differentiation,
  % circuit behavior model, MOS sizing
\end{abstract}

\newpage
\tableofcontents

\newpage
\section{引言}
模拟电路仿真(图\ref{fig:simulator-flowchart})
\cite{nagel1971computer,mccalla1971bias,Nagel:M382},经过在代数微分方程理论
\cite{estevez2000structural,kunkel2006differential,gunther2005modelling}、
器件建模\cite{chauhan2012bsim,ezaki2008physics,gildenblat2006psp}、
方程组构建\cite{hachtel1971sparse,ho1975modified,fijnvandraat2002time}、
方程组求解\cite{nastov2007fundamentals,najm2010circuit}、
硬件描述语言(HDL)\cite{verilog2014verilog,ieee2006ieee-1364-2005,
lemaitre2002adms,christen1999vhdl,pecheux2005vhdl}
等各领域的长期发展,已成为模拟电路 EDA 工具链中辅助验证和设计最重要的工具之一。
\begin{figure}[htpb]
  \centering
  \includegraphics[width=1.0\textwidth]{fig/simulator-flowchart.pdf}
  \caption{仿真器内部流程图}
  \label{fig:simulator-flowchart}
\end{figure}
然而,当前主流 HDL 与仿真器对于复用粗粒度电路模块行为模型、引入多物理效应
以及在自动设计中应用梯度优化方法等方面的功能支持还有一些不便之处:
\begin{itemize}[partopsep=0pt,topsep=0pt,itemsep=0pt,parsep=0pt]
  \item
    器件及电路建模方面,HDL 常被用于实现器件行为模型,再编译
    \cite{lemaitre2002adms}为仿真器可调用的程序。 然而,% 引入器件 Aging 效应
    % 需借助 CMC-OMI\cite{CMC-OMI,velarde2019integration} 来修改模型参数;
    多端口电路行为建模常常需拆成两步:先做函数拟合,再表达成 HDL 实现,
    例如神经网络拟合\cite{meijer2001neural,zhang2017artificial}+HDL
    或 Volterra 多项式\cite{zhang2014large}+SPICE 网表
    等等\cite{zhang2019new,roymohapatra2019novel}。
  \item
    模拟电路设计\cite{razavi2002design,silveira1996g,jespers2009gm}
    自动化需考虑器件尺寸寻优,使电路多指标达到较优。
    在早期,研究者可借助参数敏感度分析使用梯度优化
    \cite{zhan2004optimization,agrawal2006circuit}求解。
    随着工艺及软件演进,更改设计变量需借助软件平台触发 Callback
    以修改模型参数,这导致用户无法获取变量的梯度信息,
    因此,新的研究热点转向了黑盒方法,如局部采样重构梯度
    \cite{huang2013efficient,nieuwoudt2005multi,peng2016efficient}、
    建立代理模型
    \cite{girardi2011analog,lyu2018batch,wang2014enabling,lyu2017efficient}
    或强化学习\cite{tang2018parametric}。
\end{itemize}
对上述现状,有一些模型编译、梯度获取方面的工作,例如 \citet{mahmutoglu2018new}
开发的 Verilog-AMS 编译器可使用 Matlab/Octave 运行,\citet{kuthe2020verilogae}
在编译 Verilog-AMS 电路模块时可获得更多方程及内部导数信息,
\citet{hu2020adjoint} 研究了瞬态仿真伴随方程的一种高效实现。
但是,这些工作均未考虑从方程组构建方法层面提供新的功能支持。

事实上,造成这些不便的一个重要原因是,
模拟 HDL 为同时支持结构信息与行为信息的表示,
包含了许多复杂乃至臃肿的功能,
例如模拟信号仿真所必须的自动微分、
建模可能用到的插值拟合\cite[Sec4.5.6,Sec9.21]{verilog2014verilog},
这两种信息的耦合导致模拟 HDL 既不易受益于其他语言及工具的开放生态,
也增加了开发模拟 EDA 工具的门槛。
此外,嵌套电路模块之间仅可传递不依赖于信号值的静态电路参数,
仿真运行时变量只能存在于模块内
(\cite[Sec3.4,Sec6]{verilog2014verilog},\cite[Sec4.10]{ieee2006ieee-1364-2005}),
% \cite[Sec3.4,Sec6; Sec4.10]{verilog2014verilog,ieee2006ieee-1364-2005},
这不利于粗粒度电路模型的复用与开发。

\paragraph{本文工作} 近年来,深度学习\cite{goodfellow2016deep}及
自动微分编程\cite{baydin2018automatic}等优秀工具的发展,
影响了科学计算中数据驱动的多尺度建模、反问题等许多研究
\cite{zhang2018deep,long2018pde,long2019pde}。
受此启发,本文提出了一种层次化(Hierarchical Circuit Simulation)
电路仿真\cite{fijnvandraat2002time,
ter1999numerical,mukherjee1999hierarchical,tcherniaev2003transistor}
中方程组构建器的计算图实现及对应的 JSON 网表与编译方法(图\ref{fig:flowchart}),
以统一的方式处理电路模块内外变量、设计变量、模型输入参数及相应的梯度。
本工作以电路模块作为计算图的基本计算单元,
支持使用 “JSON 网表定义等效电路分解 + SubModel 计算动态参数” 解耦表示电路模型:
(1)JSON 格式易解析,且其基本数据类型 “字典+列表” 已足以表示电路结构信息;
(2)SubModel 实现时可借助相应工具如 Julia\cite{Bezanson_Julia_A_fresh_2017}
的自动微分能力; 这种表示可降低电路模型开发难度并增强仿真工具的梯度获取能力。
\begin{figure}[htpb]
  \centering
  \includegraphics[width=0.7\textwidth]{fig/flowchart.pdf}
  \caption{计算图及JSON网表编译}
  \label{fig:flowchart}
\end{figure}

类似于编程语言的函数的定义、编译、表示、执行
\cite{aho2007compilers,muchnick1997advanced,appel2004modern},
Sec \ref{sec:Joanna} 详细讨论了计算图中电路模块结构信息的处理:
模块定义(网表)、解析与编译、模块实例的数据结构、
执行引擎(计算图,图\ref{fig:equations-system-constructor}),
此外,Sec \ref{subsec:EvalCompositeSubCkt} 还介绍了如何定义和使用 SubModel
计算内部变量作为动态参数以提供电路模块的行为信息。
Sec \ref{sec:applications} 介绍了两个应用示例,
包括器件模型、多 PVT 下联合求解 DC,AC 分析与器件尺寸寻优。

\section[Hierarchical Circuit Equations System Constructor: Computational Graph]{Hierarchical Circuit Equations System Constructor: \\ Computational Graph}\label{sec:Joanna}
\begin{figure}[htpb]
  \centering
  \begin{subfigure}{0.49\textwidth}
    \includegraphics[width=\textwidth]{fig/static-engine.pdf}
    \caption{现有技术:静态参数}
    \label{fig:static-engine}
  \end{subfigure}
  \begin{subfigure}{0.49\textwidth}
    \includegraphics[width=\textwidth]{fig/computational-graph.pdf}
    \caption{计算图:动态参数}
    \label{fig:computational-graph}
  \end{subfigure}
  \caption{层次电路方程组构建器:
  现有技术\ref{fig:static-engine} v.s. 计算图\ref{fig:computational-graph}。
  每个计算单元({\color{yellow}黄色模块})对应一个子电路,
  可逐层分解为更小的子电路,其中最小粒度子电路为“基本器件”。
  $\bm{e}^{(\cdots)}$ 表示对应子电路的内外节点,$\bm{x}$ 是广义信号,
  如节点偏置电压、支路电流;
  $\bm{p}^{(\cdots)}$ 表示该子电路涉及的输入参数,如器件尺寸,
  对于动态参数而言还可以是基本器件的非线性电容、电感、电流值等;
  $\bm{f}^{(\cdots)}$ 是子电路对仿真方程余项的贡献,
  示意图中省略了 Jacobian 矩阵的计算。
  现有技术中,电路模块下层参数不依赖于信号值$\bm{x}[\bm{e}]$,
  可在网表编译期完成 ParamExpr 的计算;
  计算图中,动态参数来自 SubModel 输出的{\color{capri}内部变量},
  在方程组计算的运行时获得。
  }
  \label{fig:equations-system-constructor}
\end{figure}
% \subsection{符号 \& 背景}
给定 $N$ 个广义信号(即方程未知数) $\bm{x}\in\mr^N$
及 $M$ 个与信号无关的输入变量或参数 $\bm{p}\in\mr^M$,
数学上,电路仿真本质是“构建+求解”如下守恒型代数微分方程
\cite{najm2010circuit,gunther2005modelling,hu2020adjoint}
\begin{equation}\label{eq:flat-equation}
  \bm{f}(\bm{\dot{x}}(t),\bm{x}(t),\bm{p})
  \triangleq\frac{\ud\bm{Q}(\bm{x},\bm{p})}{\ud t}+\bm{F}(\bm{x},\bm{p})
  =\bm{0},\tag{Eq.(Flat)}
\end{equation}
其中,$\bm{Q}$ 是方程余项的动态部分,例如电容器的电荷、电感器的磁通量;
$\bm{F}$ 是方程余项的静态部分,如节点总流入直流电流,电压源压降方程。
% Hierarchical 仿真: Cadence UltraSim, Synopsys HSPICE;
现代仿真器通常将 \ref{eq:flat-equation} 按照电路层次进行分解和构建,
具有更易理解、并行的优点:
\begin{equation}\label{eq:hierarchical-equation}
  \begin{split}
    \bm{f}(\bm{\dot{x}},\bm{x},\bm{p})
    & = \bm{f}^{(1)}(\bm{\dot{x}}^{(1)},\bm{x}^{(1)},\bm{p}^{(1)})
    + \bm{f}^{(2)}(\bm{\dot{x}}^{(2)},\bm{x}^{(2)},\bm{p}^{(2)})
    + \cdots \\
    & = \bm{f}^{(1,1)}+\bm{f}^{(1,2)}+\cdots+\bm{f}^{(2,1)}+\bm{f}^{(2,2)}+\cdots, \\
    & \cdots
  \end{split}\tag{Eq.(Hierarchical)}
\end{equation}
其中 $\bm{x}^{(i)},\bm{p}^{(i)},\bm{f}^{(i)}$ 分别是原电路的
第 $i$ 个子电路的输入信号、参数或变量及贡献的方程余项,
$\bm{x}^{(i)},\bm{f}^{(i)}$ 之间相互交叠的部分取决于各子电路公共节点。
如 \ref{eq:hierarchical-equation} 所示,$\bm{f}^{(i)}$ 也可按需进一步
逐级分解为 $\bm{f}^{(i,1)},\bm{f}^{(i,2)},\cdots$ 等。

需要说明的是,\ref{eq:flat-equation} 仅表示瞬态(TRAN)分析方程,数值求解时
常常将 \ref{eq:flat-equation} 沿时间方向离散,并在每个时间步使用如 Newton-Raphson
方法求解代数方程组
\cite[Sec 7.1]{fijnvandraat2002time}
\[
  \text{ Solve }\bm{x},\text{ Subject to }
  \frac{1}{\beta\Delta t}\bm{Q}(\bm{x},\bm{p})+F(\bm{x},\bm{p})+\bm{b}=\bm{0},
\]
需反复计算 $\bm{Q},\bm{F}$ 及稀疏 Jacobian 矩阵
$\nabla_{\bm{x}}\bm{Q},\nabla_{\bm{x}}\bm{F}$。 若考虑其他如直流(DC)分析,
交流(AC)小信号分析,则需对 \ref{eq:flat-equation} 进行转换
(附录\ref{appendix:TRAN-to-AC-equation})。 由于各分析对应方程的处理是类似的,
下面主要以TRAN分析和一个简单的 JSON 网表子电路定义
(代码块\ref{lst:size-dependent-resistor})
为例,着重说明在层次电路仿真中,如何将
$\bm{Q}^{(\cdots)},\bm{F}^{(\cdots)}$,
$\nabla_{\bm{x}^{(\cdots)}\text{ or }\bm{p}^{(\cdots)}}\bm{Q}^{(\cdots)}$,
$\nabla_{\bm{x}^{(\cdots)}\text{ or }\bm{p}^{(\cdots)}}\bm{F}^{(\cdots)}$
的计算过程表达为计算图(图\ref{fig:computational-graph})的前传和反传。


\subsection{JSON网表中的子电路模块定义}
\label{subsec:subckt-module-definition}
类似于 Verilog-A/MS\cite[Sec 6]{verilog2014verilog},
定义一个电路模块,应该包含五部分信息(表\ref{tab:subckt-module-definition}):
(1)外部节点;(2)内部节点;(3)输入参数;(4)内部子电路分解;(5)内部变量。
\begin{lstlisting}[language=json,basicstyle=\small,numbers=none,
caption={网表中自定义名为 “SizeDepResistor” 的子电路:由尺寸决定阻值的电阻器},
label=lst:size-dependent-resistor]
"SizeDepResistor":{ # 定义子电路模块
  “ExternalNodes":["l","r"],
  “InputParams":["Rlength","Rwidth"],
  “InternalNodes":[],
  “SubModel":{
    “Expr":"[1e2*Rlength/Rwidth,]",
    “IntrinsicParams":["RValue"]
  },
  “Schematic":{ # 模块内各子电路/器件实例化语句
    “instanceR":{
      “MasterName":"resistor",
      “ExternalNodes":{"left":"l","right":"r"},
      “InputParams":{"resistance":"RValue"}
    }
  }
}
\end{lstlisting}
\begin{table}[htbp]
  \centering
  \caption{子电路模块定义}\label{tab:subckt-module-definition}
  \begin{tabular}{l|l|l|l}
    \hline
    & \multicolumn{1}{c|}{内容} & \multicolumn{1}{c|}{字段} & \\
    \hline
    \multirow{4}{*}{\begin{tabular}[c]{@{}r@{}}结构信息\\{\small(字典+列表)}\end{tabular}}
    & 外接节点名列表 & ExternalNodes & 必须 \\
    & 内部节点名列表 & InternalNodes & 必须 \\
    & 输入参数名列表 & InputParams   & 必须 \\
    & 内部子电路分解 & Schematic     & 必须 \\
    \hline
    \begin{tabular}[c]{@{}r@{}}行为信息\\{\small(可微函数映射)}\end{tabular}
    & 内部变量子模型 & SubModel      & 可选 \\
    \hline
\end{tabular}
\end{table}
其中,“Schematic”字段表示内部子电路分解,其中包含零或若干个子电路/器件的实例化语句,
每个实例化语句的组成是 (1)实例名;(2)类型名;(3)对外节点连接;(4)输入参数值。
以代码块\ref{lst:size-dependent-resistor}为例,其子电路分解中仅包含一个实例,
\begin{itemize}[partopsep=0pt,topsep=0pt,itemsep=0pt,parsep=0pt]
  \item “instanceR” 是实例名称。
  \item “MasterName” 指出该实例的类型是 “resistor”。
    实例类型可以是其他子电路模块,也可以是内部支持的基本器件类型。
  \item “ExternalNodes” 指出该实例的两个外接节点 “left”,“right” 连接在
    当前模块,即 “SizeDepResistor” 的 “l”,“r” 上。
    一般情况下,“Schematic” 中各个实例所连接的节点应当
    来自于当前模块的内外节点:“ExternalNodes” 以及 “InternalNodes”。
  \item “InputParams” 指出该实例的参数是 “SubModel” 子模型计算出来的内部变量
    “RValue”。一般情况下,“Schematic” 中实例所引用的参数来自于:
    (1)全局变量;
    (2)当前模块的“InputParams”;
    (3)当前模块 “SubModel” (如果有)下的 “IntrinsicParams”。
\end{itemize}
关于 SubModel 及其作用的更多讨论参考下文 Sec \ref{subsec:EvalCompositeSubCkt}。

\subsection{子电路模块实例在程序中的表示}
\label{subsec:subckt-instance-data-structure}
% 无需 EvalCompositeSubCkt 算法细节,梳理数据
为使方程组构建器可高效调用子电路模块,子电路定义
应当编译为合适的层次化数据结构(图\ref{fig:subckt-instance-data-structure}),
每个子电路模块定义经编译后通常包含两部分:
(1)同类子电路共享规则(表\ref{tab:BasicCompositeSubCktRule});
(2)实例私有数据(表\ref{tab:CompositeSubCkt})。
\begin{figure}[htpb]
  \centering
  \includegraphics[width=0.8\textwidth]{fig/subckt-instance-data-structure.pdf}
  \caption{层次子电路模块实例及计算规则示意。
  {\color{red}红色字体}部分是与现有技术\cite{tcherniaev2003transistor}的不同点:
  现有技术不需要支持动态参数,因此仅需在各电路模块计算规则中存储器件的固定参数;
  本方法中传递给下层实例、器件的参数均在运行时计算,因此需建立参数的索引。}
  \label{fig:subckt-instance-data-structure}
\end{figure}
\begin{table}[htbp]
  \centering
  \caption{子电路模块实例数据}\label{tab:CompositeSubCkt}
  \begin{tabular}{l|l}
    \hline
    \multicolumn{1}{c|}{符号} & \multicolumn{1}{c}{含义} \\
    \hline
    rule        & 指向该类子电路规则(表\ref{tab:BasicCompositeSubCktRule})的指针 \\
    \textbf{in} & 子电路实例的内部节点序号,\textbf{i}nternal \textbf{n}odes \\
    subckts     & 下层子电路实例指针 \\
    \hline
  \end{tabular}
\end{table}
\begin{table}[htbp]
  \centering
  \caption{子电路模块计算规则}\label{tab:BasicCompositeSubCktRule}
  \begin{tabular}{l|l}
    \hline
    \multicolumn{1}{c|}{符号} & \multicolumn{1}{c}{含义} \\
    \hline
    \textbf{c}       & 常数参数                                    \\
    \textbf{gv}      & 全局变量                                    \\
    SubModel         & 用于计算内部变量的子模型 \\
    SubCktsInfo      & 下层子电路节点、参数索引 \\
    BasicElementInfo & 基本器件节点、参数索引 \\
    \hline
  \end{tabular}
\end{table}
有几点需要说明
\begin{enumerate}[partopsep=0pt,topsep=0pt,itemsep=0pt,parsep=0pt]
  \item 各层子电路的外接节点来自上层子电路。
    其中,顶层电路应构成封闭系统,无外接节点。
  \item 不同子电路实例可能共用外部节点,但独享内部节点。
    实例化子电路时需注意使各内部节点互不冲突。
  \item 全局变量与系统信号 $\bm{x}$ 对于所有子电路模块均全局可见,
    模块计算规则(表\ref{tab:BasicCompositeSubCktRule})中仅需存储对所有全局变量
    的索引 \textbf{gv},模块内外节点也均以序号即索引形式存储及传递。
  \item 如果需要更多用于支持交互式分析及 Debug 的信息:
    可在计算规则(表\ref{tab:BasicCompositeSubCktRule})
    中添加子电路类型名、内外节点参数名、下层子电路实例化语句,
    甚至在实例(表\ref{tab:CompositeSubCkt})中动态记录输入变量
    (\textbf{i}nput \textbf{p}arameters) \textbf{ip} 等。
\end{enumerate}

\subsection{子电路模块定义到实例的编译}
\label{subsec:subckt-module-compilation}
JSON 网表文件的解析可借助程序语言的 JSON 解析工具,而子电路模块定义
(Sec \ref{subsec:subckt-module-definition})的编译分为两个步骤
\begin{enumerate}[partopsep=0pt,topsep=0pt,itemsep=0pt,parsep=0pt]
  \item 编译所有子电路模块的计算规则(图\ref{fig:compile-subckt-rule})。
    这里 SubModel 的解析与编译取决于编译器本身的实现,
    其他结构信息的处理,即节点、参数索引的建立,只需使用最基本的
    算法和数据结构如列表、字典即可完成。
  \item 递归实例化层次电路模块(图\ref{fig:cktrule-to-subckt})。
    其中,从顶层电路启动实例化程序时,输入的节点序号偏置$n=0$。
    图中所示方法保证了各个子电路模块的内部节点是相互独立的。
\end{enumerate}
\begin{figure}[htpb]
  \centering
  \begin{subfigure}{0.59\textwidth}
    \includegraphics[width = \textwidth]{fig/compile-subckt-rule.pdf}
    \caption{编译单个模块的计算规则}
    \label{fig:compile-subckt-rule}
  \end{subfigure}
  \begin{subfigure}{0.35\textwidth}
    \includegraphics[width = \textwidth]{fig/cktrule-to-subckt.pdf}
    \caption{递归实例化}
    \label{fig:cktrule-to-subckt}
  \end{subfigure}
  \caption{电路模块编译}
  \label{fig:subckt-module-compilation}
\end{figure}
需要注意的是,电路模块内部子电路分解可同时包含下层电路模块和基本器件,
因此编译器也需识别哪些实例是基本器件,哪些实例是网表中定义的子电路模块,
并分别建立节点、参数的索引。
其他内容,如检查子电路类型是否有循环定义、网表是否包含未定义子电路模块、
电路连通性、电路中是否有未用到的节点等检查,这里不再赘述。
\paragraph{基本器件}
\addcontentsline{toc}{subsubsection}{\ \ \ \ 基本器件}
可以认为是内置支持的最小粒度子电路,没有内部节点、内部器件,
当前支持的基本器件部分列表可参考表\ref{tab:basic-elements-partial-list}。
若要新增一种基本器件,需完成的工作包括(1)定义各分析下的电学响应函数;
(2)为编译器提供外接节点、输入参数等信息。
\begin{table}[htpb]
  \centering
  \caption{部分基本器件列表}
  \label{tab:basic-elements-partial-list}
  \begin{tabular}{l|l|l|l}
    \hline
     \multicolumn{1}{c}{MasterName}   & \multicolumn{1}{|c|}{ExternalNodes} &
     \multicolumn{1}{c|}{InputParams} & 备注             \\
    \hline
     resistor   & left,right              & resistance  & 电阻器           \\
     capacitor  & input,output            & capacitance & 电容器           \\
     inductor   & input,output            & inductance  & 电感器           \\
     CS         & input,output            & current     & 电流源           \\
     VS         & input,output            & voltage     & 电压源           \\
     VCCS       & left,right,input,output & MF          & 电压受控电流源   \\
     CCCS       & iorigin,input,output    & MF          & 电流受控电流源   \\
     VCVS       & left,right,input,output & MF          & 电压受控电压源   \\
     CCVS       & iorigin,input,output    & MF          & 电流受控电压源   \\
    \hline
  \end{tabular}
\end{table}

值得说明的是,按照改进的节点分析法\cite{ho1975modified},
电压源类型的基本器件必须将支路电流也作为一个自由度,
在编译器中处理为该器件外接 GALV 节点,相应的,
编译时需在上层模块新增一个内部节点。
非电压源类基本器件如电阻器也可增加一个广义外接 GALV 节点,
该节点信号通常表示流经器件支路的电流,以 TRAN 分析为例,
记电阻器的阻值为 $R$,左右节点是 $l,r$,电压值 $x_l,x_r$,
GALV 节点(如果有)及电流值是 $i,x_i$,则电阻器对应的方程余项以稀疏向量表示是
\begin{description}
  \item[无外接 GALV:] $\bm{Q}=\bm{0},\bm{F}=[(l,-\frac{x_l-x_r}{R}),(r,\frac{x_l-x_r}{R})]$。
  \item[外接 GALV:] $\bm{Q}=\bm{0},\bm{F}=[(l,-x_i),(r,x_i),(i,x_r-x_l+R\cdot x_i)]$。
\end{description}
这实际上就是\cite[Sec 2.4.4]{najm2010circuit}中同类器件的
不同 element stamp,也可理解为同类器件的
两种网络分析方法\cite{ho1975modified,hachtel1971sparse},
这一特性需要编译器与方程组构建器同时支持。
在不同仿真分析中,基本器件仿真方程余项和梯度的计算需加以区分,这里不再详细讨论。

\subsection{执行:计算图的前传和反传}\label{subsec:EvalCompositeSubCkt}
计算图\ref{fig:computational-graph}的每个基本计算单元对应一个子电路实例。
当子电路在计算图中被调用时,计算单元
首先从上层电路输入外接节点、输入变量,然后遍历内部子电路、基本器件以计算
方程余项、信号梯度、变量梯度,最后将这些结果返回给上层。
计算单元的内部过程可表达为算法\ref{alg:EvalCompositeSubCkt}:\\
\begin{algorithm}[H]
  % \TitleOfAlgo{调用子电路}
  \caption{调用子电路\\
  方程余项,信号梯度,变量梯度 = EvalCompositeSubCkt($\bm{x}$, ckt, \textbf{en}, \textbf{ip})}
  \label{alg:EvalCompositeSubCkt}
  \SetAlgoLined
  输入:所有系统信号 $\bm{x}$,子电路实例 ckt,外接节点下标 \textbf{en},输入变量 \textbf{ip}\;
  \# ckt 内部可获得信息:内部节点 \textbf{in},子模型 SubModel,全局变量 \textbf{gv},常数 \textbf{c}\;
  1. 组装 ckt 内部节点 \textbf{in} 和外接节点 \textbf{en} 得到 \textbf{nodes}=[\textbf{en},\textbf{in}]\;
  2. 依据内外信号及输入变量计算内部变量 \textbf{intrp} = 
  SubModel($\bm{x}$[\textbf{nodes}],\textbf{ip})\;
  3. 组装 ckt 当前所有变量及参数得到 \textbf{params}=
  [\textbf{ip},\textbf{intrp},\textbf{gv},\textbf{c}]\;
  4. 从 \textbf{nodes},\textbf{params} 提取 ckt 内部各子电路 subckt
  外接节点 \textbf{suben}$\subset$\textbf{nodes}、
  输入变量 \textbf{subip}$\subset$\textbf{params},
  并调用 EvalCompositeSubCkt($\bm{x}$, subckt, \textbf{suben}, \textbf{subip})\;
  % $\bm{Q},\bm{F},\nabla_{\bm{x}}\bm{Q},\nabla_{\bm{x}}\bm{F}$,$\nabla_{\bm{gv}}\bm{Q},
  % \nabla_{\bm{subip}}\bm{Q},\nabla_{\bm{gv}}\bm{F},\nabla_{\bm{subip}}\bm{F}$\;
  5. 从 \textbf{nodes},\textbf{params} 提取 ckt 内部各基本器件外接节点、输入变量并
  计算各基本器件的方程余项、梯度\;
  6. 收集4,5步的所有方程余项\;
  7. 依据 1-5 下标映射,反传下层子电路、基本器件的信号梯度及变量梯度\;
  输出:方程余项,信号梯度,变量梯度
\end{algorithm}
\vspace{\parsep}
\begin{figure}[htpb]
  \centering
  \includegraphics[width=0.7\textwidth]{fig/EvalCompositeSubCkt.pdf}
  \caption{算法\ref{alg:EvalCompositeSubCkt} 1-5步示意图。
  图\ref{fig:computational-graph}中顶层至第一层子电路调用的放大版。
  现有技术的电路模块中,输入参数和内部参数都可以在编译器计算,
  运行时不反传参数的梯度。}
  \label{fig:EvalCompositeSubCkt}
\end{figure}
这里 \textbf{en},\textbf{ip},\textbf{in},\textbf{gv},\textbf{intrp} 分别是
external nodes, input parameters, internal nodes, global variables,
intrinsic parameters 的缩写。 电路的内外节点\textbf{en},\textbf{in}
可分别用于索引广义信号值 $\bm{x}[\textbf{en}],\bm{x}[\textbf{in}]$。
电路模块涉及的变量/参数 $\bm{p}$ 则由
\textbf{ip},\textbf{gv},\textbf{intrp},\textbf{c} 四部分所组成。
图\ref{fig:EvalCompositeSubCkt}是此算法1-5步的示意图。

\paragraph{SubModel 与内部变量}
\addcontentsline{toc}{subsubsection}{\ \ \ \ SubModel与内部变量}
计算图中,SubModel 起到的作用是,输入当前模块的内外信号及输入变量,
输出所有\textbf{内部变量} \textbf{intrp}=SubModel(\textbf{signals},\textbf{ip})
(其中 \textbf{nodes}=[\textbf{en},\textbf{in}],
\textbf{signals}=$\bm{x}[\textbf{nodes}]$),
这些内部变量可被传递给下层子电路及基本器件。
该设定对应于这样的假设\ref{assumption:intrinsic-params-dependencies}:
“电路模块的行为应由内外信号及输入变量唯一确定”,
因此对 SubModel 而言,
不需要感知下层子电路的内部信号,或其他无关模块的节点信号或参数。
这一设定可以涵盖相当范围的非线性效应,且易于编程实现。
\begin{assumption}\label{assumption:intrinsic-params-dependencies}
  子电路模块中的所有内部变量由子电路内外节点偏置信号和输入变量决定。
\end{assumption}
电路定义中 SubModel 需提供足够信息,使得编译器可将 SubModel 注册到
共享规则(表\ref{tab:BasicCompositeSubCktRule})中,
且计算图和 SubModel 之间应当有某种协议用于获取\textbf{intrp}关于
\textbf{signals},\textbf{ip}的 Jacobian 矩阵:
\vspace{-0.5em}
\begin{equation}\label{eq:submodel-jacobian}
  J_{\textbf{s}} = \nabla_{\textbf{signals}}\textbf{intrp},
  J_{\textbf{ip}}=\nabla_{\textbf{ip}}\textbf{intrp},
\vspace{-0.5em}
\end{equation}
具体实现方式取决于所用的程序语言,这里不再赘述。

\paragraph{逐层梯度反传}
\addcontentsline{toc}{subsubsection}{\ \ \ \ 逐层梯度反传}
计算图调用子电路的计算过程与通常的层次电路方程组计算的主要区别
(图\ref{fig:equations-system-constructor})是:计算图中,
由于下层模块或器件的输入参数 \textbf{subip} 是全部参数 \textbf{params} 的子集
(图\ref{fig:EvalCompositeSubCkt}),因此,还需处理输入变量的梯度的反传。
这里以 TRAN 仿真为例,着重说明在算法\ref{alg:EvalCompositeSubCkt}中如何反传梯度。

对 TRAN 分析而言,算法\ref{alg:EvalCompositeSubCkt}的返回值实际包含8项:
$\bm{Q},\bm{F},\nabla_{\bm{x}}\bm{Q},\nabla_{\bm{x}}\bm{F}$,
$\nabla_{\textbf{gv}}\bm{Q},\nabla_{\textbf{ip}}\bm{Q}$,
$\nabla_{\textbf{gv}}\bm{F},\nabla_{\textbf{ip}}\bm{F}$。
由于 $\bm{Q}$ 与 $\bm{F}$ 的梯度反传没有区别,因此,下面为了简便,仅考虑 $\bm{Q}$。
算法\ref{alg:EvalCompositeSubCkt}中收集到的所有子电路计算结果记做
$\{\bm{Q}^{i}\}$,$\{\nabla_{\bm{x}}\bm{Q}^{i}\}$,
$\{\nabla_{\textbf{gv}}\bm{Q}^{i}\}$,$\{\nabla_{\textbf{subip}^i}\bm{Q}^{i}\}$,
这里上标 $i$ 表示内部子电路或基本器件的序号。
$\bm{Q}^{i}$,$\nabla_{\bm{x}}\bm{Q}^{i}$,$\nabla_{\textbf{gv}}\bm{Q}^{i}$
本身可直接组装,
\[
  \bm{Q} = \sum_i \bm{Q}^{i},
  \nabla_{\bm{x}}\bm{Q} = \sum_i \nabla_{\bm{x}}\bm{Q}^{i},
  \nabla_{\textbf{gv}}\bm{Q} = \sum_i \nabla_{\textbf{gv}}\bm{Q}^{i},
\]
而 $\nabla_{\textbf{subip}^i}\bm{Q}^{i}$ 的梯度反传则需按照 $\textbf{subip}^i$
对\textbf{params}=[\textbf{ip},\textbf{intrp},\textbf{gv},\textbf{c}]
的索引来分情况处理:
\begin{enumerate}[partopsep=0pt,itemsep=0pt,parsep=0pt]
  \item 若 $\text{subip}^i[j]\in\textbf{c}$,则无需反传。
  \item 若 $\text{subip}^i[j]\in\textbf{ip}\cup\textbf{gv}$,
    则直接反传 $\nabla_{\text{subip}^i[j]}\bm{Q}^i$
    至对应的 $\nabla_\textbf{ip}\bm{Q}$ 或 $\nabla_\textbf{gv}\bm{Q}$。
  \item 若有某个下标 $l$,使得 $\text{subip}^i[j]=\textbf{intrp}[l]$
    (图\ref{fig:EvalCompositeSubCkt})。
    注意到前面提到的假设\ref{assumption:intrinsic-params-dependencies}
    及内部变量关于信号与输入变量的 Jacobian 矩阵(式\ref{eq:submodel-jacobian}),
    令 $\bm{g}\triangleq\nabla_{\textbf{subip}^i[j]}\bm{Q}^{i}$,可得
    \vspace{-1em}
    \begin{equation}\label{eq:intrinsic-params-backward}
      \nabla_{x[\textbf{nodes}]}\bm{Q}\mathrel{+}=J_{\textbf{s}}[:,l]\otimes\bm{g},
      \nabla_{\textbf{ip}}\bm{Q}\mathrel{+}=J_{\textbf{ip}}[:,l]\otimes\bm{g}.
    \vspace{-1em}
    \end{equation}
    其中 $\otimes$ 表示两个向量的外积。
\end{enumerate}





\section{Applications}\label{sec:applications}
\subsection{CMOS 器件模型:等效电路分解+动态参数}\label{subsec:cmos-model}
如前文所述,任何符合假设\ref{assumption:intrinsic-params-dependencies}
的子电路模块,例如 CMOS 管,都可基于 SubModel 机制建模为“等效电路分解+动态参数”,
这一节我们就提供一个基于查找表实现的示例。
需说明的是,更兼容现有技术的做法应该是直接
在 SubModel 中重新实现 BSIM 模型\cite{chauhan2012bsim},这在本文不涉及。
CMOS 模块示例的具体定义可参考附录\hyperref[appendix:mos-subckt]{B},
它在 AC 分析下的等效电路图为图\ref{fig:mos-small-signal-model-2},
其定义的基本要素如下
\begin{figure}[htpb]
  \centering
  \includegraphics[width=0.7\textwidth]{fig/mos-small-signal-model-2.pdf}
  \caption{CMOS 等效小信号模型,参考\cite[Figure 2.39]{razavi2002design}。
  这里 Ro 是阻值为 $\frac{1}{\text{GDS}}$ 的电阻器。}
  \label{fig:mos-small-signal-model-2}
\end{figure}
\begin{enumerate}[partopsep=0pt,topsep=0pt,itemsep=0pt,parsep=0pt]
  \item 内外节点是 \textbf{nodes}=[gate,source,drain,bulk]。
  \item 输入变量即该器件尺寸 \textbf{ip}=[MosL,MosW]。
  \item SubModel 输出的内部变量
    \textbf{intrp}=[ID,GDS,CDD,CSS,CGG,CGS,CGD,GM,GMB]
    值由 \textbf{nodes} 四端偏置电压及器件尺寸\textbf{ip}决定。
\end{enumerate}
编译器读取该模块定义后,
将加载外部库并生成一个 “lut.MosLookup” 类型的函数对象作为 SubModel 注册到
子电路规则(表\ref{tab:BasicCompositeSubCktRule})中。
内部变量 ID 表示 DC 分析下 source,drain 之间的直流电流,
GDS,CDD,GM 等则是 AC 分析下的等效小信号参数,其中
\begin{enumerate}[partopsep=0pt,topsep=0pt,itemsep=0pt,parsep=0pt]
  \item 内置基本器件 ICS,ACVCCS (附录\hyperref[appendix:mos-subckt]{B})
    的作用是使得 ID 仅在 DC 分析下起作用,
    而 GDS,GM,GMB 仅在 AC 分析下起作用。
  \item 方程构建器的 DCAC 混合分析或 DC 分析计算图可运行此电路模块,
    但 AC 分析计算图不可单独运行此电路模块:
    要建立 AC 分析方程需先计算由 DC 偏置电压决定的 [GDS,GM] 等,
    这与直接通过 TRAN 分析方程诱导出小信号线性方程不同
    (附录\hyperref[appendix:TRAN-to-AC-equation]{A})。
    事实上,假设\ref{assumption:intrinsic-params-dependencies} 中也规定,
    内部变量可依赖于偏置电压信号,而不可依赖于线性分析中的小信号。
  \item SubModel 可在确保遵循相应自动微分系统的接口要求的前提下,
    自由调用外部程序,例如使用三维样条插值。
\end{enumerate}
从示例说明中可以看出,基于 SubModel 机制的 “等效电路分解+动态参数”
器件模型表示法具有如下优势
\begin{enumerate}[partopsep=0pt,topsep=0pt,itemsep=0pt,parsep=0pt]
  \item SubModel 与电路的网络分析、仿真是相互解耦的,它只负责
    内部变量及 Jacobian 矩阵的计算(Sec \ref{subsec:subckt-instance-data-structure}),
    无需关心电路连接关系。
  \item SubModel 计算内部变量的语法及能力边界取决于编译器对网表中
    “SubModel”字段的处理,实现上可考虑借助各类外部程序及自动微分工具。
\end{enumerate}
\subsection{运算放大器尺寸寻优:多 PVT 下器件DC工作状态优化}
模拟电路设计的尺寸寻优流程如图\ref{fig:manually-design},
设计师将目标工艺提供的可用器件连接成各类电路,
并基于一定的方法论和经验调节器件如 CMOS 管的尺寸(即Device Sizing)
使得电路可在给定面积、功耗约束下达成各类指标,
这期间需反复使用电路仿真来定量了解电路行为及性能,而无需实际制造。
\begin{figure}[htpb]
  \centering
  \includegraphics[width=\textwidth]{fig/manually-design.pdf}
  \caption{尺寸寻优:人工迭代流程}
  \label{fig:manually-design}
\end{figure}

这个流程可自然的转换成优化问题来求解,
根据梯度是否可获取,可采取不同的优化策略
\cite{zhan2004optimization,agrawal2006circuit,huang2013efficient,
nieuwoudt2005multi,peng2016efficient,girardi2011analog,lyu2018batch,
wang2014enabling,lyu2017efficient,tang2018parametric}。
为了获取梯度,以 DC 仿真为例:直流稳态方程组
$\bm{F}(\bm{x},\bm{p})=\bm{0}$ 自然地给出了参数 $\bm{p}$
到方程组的解 $\bm{x}^{solution}$ 的隐式映射,该映射的 Jacobian 矩阵为
$\nabla_{\bm{p}}\bm{x}^{solution}=-\nabla_{\bm{x}}\bm{F}\backslash\nabla_{\bm{p}}\bm{F}$,
而 $\nabla_{\bm{x}}\bm{F},\nabla_{\bm{p}}\bm{F}$ 可由 Sec \ref{sec:Joanna}
介绍的方程组构建方法(计算图\ref{fig:computational-graph}) 直接给出。
借助这些信息,我们可以使用梯度优化方法(图\ref{fig:solve-then-optimize})。
注意,实际优化过程中,无需完整求解 $\nabla_{\bm{x}}\bm{F}$ 的逆,
只需要对给定的损失函数或约束函数 $l$,在每次迭代的梯度反传时求解线性方程组即可:
$\nabla_{\bm{p}}l(\bm{x}^{solution})=(\nabla_{\bm{p}}\bm{x})^T\cdot\nabla_{\bm{x}}l
=-(\nabla_{\bm{p}}\bm{F})^T\cdot\big(\nabla_{\bm{x}}\bm{F}^T\backslash\nabla_{\bm{x}}l\big)$
\begin{figure}[htpb]
  \centering
    \includegraphics[width = 0.7\textwidth]{fig/solve-then-optimize.pdf}
  \caption{尺寸自动寻优}
  \label{fig:solve-then-optimize}
\end{figure}

\begin{figure}[htbp]
  \centering
  \begin{subfigure}{0.3\textwidth}
    \includegraphics[width=\textwidth]{fig/amplifier-diagram.png}
    \caption{运算放大器示意图\cite{OpAmpPNG}}
    \label{subfig:amplifier-diagram}
  \end{subfigure}
  \begin{subfigure}{0.65\textwidth}
    \includegraphics[width=\textwidth]{fig/amplifier-transfer-function.png}
    \caption{尺寸优化后,运放的输入输出频响曲线}
    \label{subfig:amplifier-transfer-function}
  \end{subfigure}
  \caption{\textbf{左(a)}:运算放大器示意图,内部包含偏置电路及主电路
  共17个n-MOSFET和17个p-MOSFET。
  在 DC 工作点下,可以在 $V_+,V_-$ 输入角频率 $\omega$ 的扰动小信号,
  从$V_{out}$处可以检测到输出信号。 \textbf{右(b)}:尺寸优化后,
  运放在 $Corner=tt,Temperature=27$
  工作条件下的频响曲线。 其中 va4,va5 是电路中的两个内部节点。}
\end{figure}
以一个运算放大器(图\ref{subfig:amplifier-diagram})为例,我们把该电路中各设计变量
即所有 MOS 管的沟道长和宽作为待优化变量。
给定外部电流电压源 Ibias0,Ibias1 及 $V_{dd},V_+,V_-$,以及负载电阻电容
$R_L=200\Omega,C_L=10^{-10}\text{F}$,这里我们设置的指标如下,
\begin{enumerate}[partopsep=0pt,topsep=0pt,itemsep=0pt,parsep=0pt]
  \item 在 $Corner\in[tt,ff,ss],Temperature\in[27,-40,125]$ 共 9 种工作条件下,
    要求所有 MOS 管的 DC 工作状态均为饱和,以 NMOS 管为例,其 DC 偏置电压应满足
    \vspace{-1em}
    \[
      \min(V_{gs},V_{ds},V_{sb},V_{gs}-V_{th})\geq0.
      \vspace{-1em}
    \]
    其中 $V_{th}$ 依赖于 MosL,$V_{gs},V_{ds}$。
    对不同工作条件 SubModel 需加载不同的数据库,最终得到的仿真解也各有区别。
  \item 在典型工作条件 $Corner=tt,Temperature=27$ 下,允许 $V_+,V_-$
    电压源有上下浮动,但需满足关系 $V_++V_-=5v$,
    此时,$V_{out}$ 的DC 偏置工作点作为 $V_+,V_-$ 的函数,
    要求其最大值大于 4.35v,最小值小于 0.3v,
    这一要求类似于电路的输出摆幅指标。
  \item 在典型工作条件 $Corner=tt,Temperature=27$ 下,对电路进行 AC 分析,
    在 $V_+,V_-$ 处输入 $v_{in+},v_{in-}=\pm0.5$ 的信号,
    要求 $V_{out}$ 的直流增益 $gain=20\cdot\log_{10}(|v_{out}|)$ 达到100。
  \item 各器件设计变量满足事先规定的对称性约束,例如输入对管
    $M_{n0\_in},M_{n0\_ip}$ 的尺寸 MosL,MosW 应完全相等,电流境
    $M_{p30\_mirr},M_{p20\_mirr},M_{p10\_mirr},M_{p50\_mirr},M_{p60\_mirr}$
    的沟道长 MosL 应相等,等等。
\end{enumerate}
\begin{equation}\label{eq:optimization}
  \begin{split}
    \min_{\bm{p}} l &= \max(5-\log_{10}(|\bm{v}[out]|),0)^2 \\
    \st \ \ \ \ \ \ 
    & \forall c\in[tt,ff,ss],t\in[27,-40,125], \\
    & \ \ \ \ \bm{x}_L\preceq\bm{x}^{c,t}\preceq\bm{x}_U;
    Saturation(\bm{x}^{c,t},\bm{p})\succeq\bm{0}; \\
    & \bm{x}^{down}[out]\leq0.3;\bm{x}^{up}[out]\geq4.35; \\
    & \bm{v}=\bm{A}^{tt,27}\backslash\bm{b}^{tt,27}; \ C\cdot\bm{p}=\bm{0}.
  \end{split}
\end{equation}
我们将上述设计任务表述为约束优化问题 Prob.\eqref{eq:optimization}。
其中 $\bm{p}\to\{\bm{x}^{c,t}\},\bm{x}^{down},\bm{x}^{up}$
通过求解对应工作条件及输入偏置 $V_+,V_-$ 下的 DC 方程组得到;
$\bm{v}=\bm{A}^{tt,27}\backslash\bm{b}^{tt,27}$ 是求解
$Corner=tt,temperature=27$ 时 AC 分析线性方程组,
其中的系数实际上就是依赖于 $\bm{x}^{tt,27},\bm{p}$ 的各器件的
GM,GDS 等等。
我们可借助 DCAC 混合分析的计算图来计算
$\bm{A},\bm{b},\nabla_{\bm{x}}\bm{A},\nabla_{\bm{x}}\bm{b}$
\footnote{$A=i\omega\cdot\nabla_{\bm{x}}\bm{Q}+\nabla_{\bm{x}}\bm{F}$,
但这里使用 DCAC 计算图而非 $\nabla\bm{Q},\nabla\bm{F}$ 来计算 $\nabla A$,
否则需计算和反传 $\bm{Q},\bm{F}$ 关于 $\bm{x}$ 的二阶导,
这将大大增加计算图实现难度},
进一步可获得 $\nabla_{\bm{x}}l,\nabla_{\bm{p}}l$
(参考附录\hyperref[appendix:inv-linear-equation-grad]{C});
$C\cdot\bm{p}=\bm{0}$ 表示对设计变量的直接约束,例如对称性约束。

优化算法调用开源软件 Ipopt\cite{wachter2006implementation} 实现,
一共包括 72 个待求解变量,27 个等式约束,308 个不等式约束。
整个过程(含 Julia 代码编译加载,网表解析等)在
Intel(R) Core(TM) i7-8700 CPU @ 3.20GHz 使用 6 线程耗时 356 秒,
图\ref{subfig:amplifier-transfer-function} 给出了优化后的电路频响曲线。
从实验结果可以看到,
\begin{enumerate}[partopsep=0pt,topsep=0pt,itemsep=0pt,parsep=0pt]
  \item 基于计算图的层次电路仿真或尺寸寻优,可使得器件模型、求解算法之间
    相互解耦且可灵活的通信。
  \item 计算图中将参数变量化处理,使得许多指标的梯度优化更为简单自然。
\end{enumerate}
最后需要说明的是,上述实验仅考虑了各器件的工作状态、电路在典型工作条件下的 DC 增益,
若需进行完整的设计,需在优化问题中引入更多的指标,甚至可能包含离散值指标,
各类指标整合到优化的框架中也并非一蹴而就,这些问题在这里不再详细讨论。


\section{Conclusion}
In this paper, the static parameters of the circuit are processed as runtime variables in simulation, and the structural information and behavioral information of the circuit module/device are decoupled as "equivalent subcircuit decomposition + submodel-computed dynamic parameters". These further derive the computational graph representation of the equations system constructor for hierarchical circuits with circuit modules as the compute units of the computational graph. According to the two simple examples, this approach facilitates the decoupling and flexible interaction between netlists, models, and simulation and optimization algorithms. However, some problems exist with this approach. For example, because the variable gradient will be passed across the layers of a circuit, the topology analysis for circuit equations solvability and DAE-Index no longer works, requiring a more general hierarchical analysis theory. In the future, this approach will gain better generalization by supporting BSIM and more simulation types with more effects (i.e., S parameter or thermal effect) considered. Faster simulation is also possible if the program itself is optimized and the support for fast-SPICE technology is added.

\section*{Acknowledgment}
\addcontentsline{toc}{section}{Acknowledgment}
I would like to thank my colleagues at HiSilicon. It's been both a pleasure and an honor to work with them. This work has been also supported by many EDA and HDL experts, including Waisum Wang, Jiewen Fan, Zuochang Ye, Yang Lu, Shangxia Fang, Guoyong Shi, and Yu Ji. Special thanks go to Zhenya Zhou, Jiawei Zhuang, Long Wang, and Yuwei Fan for their kind help.

\addcontentsline{toc}{section}{References}
\bibliographystyle{unsrtnat}
\bibliography{ref}
\section*{附录}
\addcontentsline{toc}{section}{附录}
\renewcommand\thesubsection{\Alph{subsection}}
\subsection{TRAN 分析诱导 AC 分析方程}\label{appendix:TRAN-to-AC-equation}
求解瞬态方程 \ref{eq:flat-equation} 及直流稳态方程 $\bm{F}(\bm{x},\bm{p})=\bm{0}$ 时,
Newton–Raphson 方法需反复计算方程余项 $\bm{Q},\bm{F}\in\mr^N$ 及 Jacobian 矩阵
$\nabla_{\bm{x}}\bm{Q},\nabla_{\bm{x}}\bm{F}\in\mr^{N\times N}$。
对交流小信号分析而言,考虑在稳态解附近对 $\bm{x},\bm{p}$ 作扰动
$\delta\bm{x},\delta\bm{p}$ 并线性化 \ref{eq:flat-equation} 得到
\[
  \nabla_{\bm{x}}\bm{Q}\cdot\dot{\delta}\bm{x}
  +\nabla_{\bm{p}}\bm{Q}\cdot\dot{\delta}\bm{p}
  +\nabla_{\bm{x}}\bm{F}\cdot\delta\bm{x}
  +\nabla_{\bm{p}}\bm{F}\cdot\delta\bm{p}=\bm{0}
\]
若假设$\delta\bm{x},\delta\bm{p}$是角频率为 $\omega$ 的小信号
$\bm{\epsilon}_{\bm{x}}\cdot e^{i\omega t},\bm{\epsilon}_{\bm{p}}\cdot e^{i\omega t}$,
则得到交流小信号分析的线性方程组
\begin{equation}\label{eq:flat-ac-equation}
  (i\omega\cdot\nabla_{\bm{x}}\bm{Q}+\nabla_{\bm{x}}\bm{F})\cdot\bm{\epsilon}_{\bm{x}}
  = (i\omega\cdot\nabla_{\bm{p}}\bm{Q}+\nabla_{\bm{p}}\bm{F})\cdot\bm{\epsilon}_{\bm{p}}
\end{equation}

\subsection{CMOS 子电路}\label{appendix:mos-subckt}
\begin{lstlisting}[language=json,numbers=none,
caption={CMOS 子电路},label=lst:mos-subckt]
"NMOSTYPE":{
  "ExternalNodes":["gate","source","drain","bulk"],
  "InputParams":["MosL","MosW"],
  "InternalNodes":[],
  "SubModel":{
    "Analysis":["DC","TRAN"],
    "ModelLoader":"SimInfo->lut.MosLookup(\"NMOSTYPE\",
      /path/to/data; SimInfo=SimInfo)",
    "IntrinsicParams":
      ["ID","GDS","CDD","CSS","CGG","CGS","CGD","GM","GMB"]
  },
  "Schematic":{
    "ids":{
      "MasterName":"ICS",
      "ExternalNodes":{"input":"source","output":"drain"},
      "InputParams":{"dc":"ID","ac":0}
    },
    "template":{
      "MasterName":"MosSmallSignalTemplate",
      "ExternalNodes":{
        "gate":"gate","source":"source",
        "drain":"drain","bulk":"bulk"
      },
      "InputParams":{
        "GDS":"GDS","CDD":"CDD","CSS":"CSS","CGG":"CGG",
        "CGS":"CGS","CGD":"CGD","GM":"GM","GMB":"GMB"
      }
    }
  }
}
\end{lstlisting}
\begin{lstlisting}[language=json,numbers=none,
caption={MosSmallSignalTemplate:小信号等效电路分解},label=lst:mos-small-signal-subckt]
"MosSmallSignalTemplate":{
    "ExternalNodes":["gate","source","drain","bulk"],
    "InputParams":["GDS","CDD","CSS","CGG","CGS","CGD","GM","GMB"],
    "InternalNodes":[],
    "Schematic":{
        "infr":{
            "MasterName":"resistor",
            "ExternalNodes":{"left":"drain","right":"source"},
            "InputParams":{"resistance":1e1000}
        },
        "gds":{
            "MasterName":"ACVCCS",
            "ExternalNodes":{"left":"drain","right":"source","input":"drain","output":"source"},
            "InputParams":{"MF":"GDS"}
        },
        "cdd":{
            "MasterName":"capacitor",
            "ExternalNodes":{"input":"drain","output":"bulk"},
            "InputParams":{"capacitance":"CDD"}
        },
        "css":{
            "MasterName":"capacitor",
            "ExternalNodes":{"input":"source","output":"bulk"},
            "InputParams":{"capacitance":"CSS"}
        },
        "cgg":{
            "MasterName":"capacitor",
            "ExternalNodes":{"input":"gate","output":"bulk"},
            "InputParams":{"capacitance":"CGG"}
        },
        "cgs":{
            "MasterName":"capacitor",
            "ExternalNodes":{"input":"gate","output":"source"},
            "InputParams":{"capacitance":"CGS"}
        },
        "cgd":{
            "MasterName":"capacitor",
            "ExternalNodes":{"input":"gate","output":"drain"},
            "InputParams":{"capacitance":"CGD"}
        },
        "gm":{
            "MasterName":"ACVCCS",
            "ExternalNodes":{
                "left":"gate","right":"source","input":"drain","output":"source"
            },
            "InputParams":{"MF":"GM"}
        },
        "gmb":{
            "MasterName":"ACVCCS",
            "ExternalNodes":{
                "left":"bulk","right":"source","input":"drain","output":"source"
            },
            "InputParams":{"MF":"GMB"}
        }
    }
}
\end{lstlisting}

\subsection{线性方程组解的梯度反传}\label{appendix:inv-linear-equation-grad}
考虑关于 $\bm{v}$ 的实线性方程组 $\bm{A}(\bm{x})\bm{v}=\bm{b}(\bm{x})$,
其中 $\bm{A},\bm{b}$ 非线性依赖于 $\bm{x}$ 为稀疏矩阵/向量,且
$\nabla_{\bm{x}}\bm{A},\nabla_{\bm{x}}\bm{b}$
均可计算,则 $\bm{v}$ 也将非线性依赖于 $\bm{x}$,事实上,对方程组作微分可得
\[
  (\bm{A}+\nabla_{\bm{x}}\bm{A}\cdot\ud\bm{x})\cdot
  (\bm{v}+\nabla_{\bm{x}}\bm{v}\cdot\ud\bm{x})
  =\bm{b}+\nabla_{\bm{x}}\bm{b}\cdot\ud\bm{x}
\]
舍弃0阶及2阶项,可得下式对任意 $\ud\bm{x}$ 成立
\begin{equation}\label{eq:inv-full-differentiation}
  \nabla_{\bm{x}}\bm{A}\cdot\ud\bm{x}\cdot\bm{v}
  +\bm{A}\cdot\nabla_{\bm{x}}\bm{v}\cdot\ud\bm{x}
  =\nabla_{\bm{x}}\bm{b}\cdot\ud\bm{x},
\end{equation}
假设某个损失函数 $l(\bm{v})$ 可关于 $\bm{v}$ 求梯度 $\nabla_{\bm{v}}l$,
需将梯度反传到 $\bm{x}$,即需求解
$\nabla_{\bm{x}}l=\nabla_{\bm{v}}l\cdot\nabla_{\bm{x}}\bm{v}$。
我们无需真的计算 $\nabla_{\bm{x}}\bm{v}$ 并存下来(这可能是稠密的),
事实上由 Eq.\ref{eq:inv-full-differentiation} 有
\[
  \begin{split}
  & \nabla_{\bm{x}}\bm{v}\cdot\ud\bm{x}
  = \bm{A}^{-1}\cdot\big(\nabla_{\bm{x}}\bm{b}\cdot\ud\bm{x}-
  \nabla_{\bm{x}}\bm{A}\cdot\ud\bm{x}\cdot\bm{v}\big),\forall \ud\bm{x}, \\
  \Rightarrow
    & \nabla_{\bm{v}}l\cdot\nabla_{\bm{x}}\bm{v}\cdot\ud\bm{x}
    = (\nabla_{\bm{v}}l\cdot\bm{A}^{-1})\cdot\big(\nabla_{\bm{x}}\bm{b}\cdot\ud\bm{x}-
  \nabla_{\bm{x}}\bm{A}\cdot\ud\bm{x}\cdot\bm{v}\big),\forall \ud\bm{x}, \\
  \end{split}
\]
因此,为计算 $\nabla_{\bm{v}}l\cdot\nabla_{\bm{x}}\bm{v}$,
只需提前作一次稀疏矩阵线性方程组求解 $\nabla_{\bm{v}}l\cdot\bm{A}^{-1}$,
再对 $\ud\bm{x}$ 各位置逐个置为1,其余位置置0,即可得到
$\nabla_{\bm{x}}l=\nabla_{\bm{v}}l\cdot\nabla_{\bm{x}}\bm{v}$。

对于复数线性方程组情形,也可采用 Wirtinger 导数进行类似讨论。

\end{document}



